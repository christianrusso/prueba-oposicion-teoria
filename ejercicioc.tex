\section{Ejercicio}
\subsection{Enunciado}
\frame{
\frametitle{Ejercicio 1-c) (Práctica 1 - Teoría de lenguajes)}
Construir un autómata finito para el siguiente lenguaje:

Cadenas sobre $\Sigma = \lbrace 0, 1 \rbrace$ con cantidad par de ceros y cantidad impar de unos.
}

\subsection{Consideraciones}
\frame{
Existen cuatro posibles combinaciones de paridades de ceros y unos para un número binario:
\begin{itemize}
\item Cantidad par de ceros, cantidad par de unos. 
\pause

Ej: $\lambda$, 0011, 1111, 010010 
\pause
\item Cantidad impar de ceros, cantidad par de unos. 
\pause

Ej: 011, 11011, 0
\pause
\item Cantidad impar de ceros, cantidad impar de unos.
\pause

Ej: 01, 1110, 0001
\pause
\item Cantidad par de ceros, cantidad impar de unos. 
\pause

Ej: 001, 111, 01000
\end{itemize}
De estas cuatro posibles combinaciones, nuestro lenguaje acepta sólo la última.
}

\subsection{Solución}
\frame{
Podemos representar las cuatro distintas combinaciones mencionadas anteriormente, como cuatro estados distintos:
\begin{itemize}
\item $Par0Par1$: Cantidad par de ceros, cantidad par de unos. 
\item $Impar0Par1$: Cantidad impar de ceros, cantidad par de unos. 
\item $Impar0Impar1$: Cantidad impar de ceros, cantidad impar de unos.
\item $Par0Impar1$: Cantidad par de ceros, cantidad impar de unos. 
\end{itemize}
}

\frame{
\textbf{Estado Inicial del autómata:}
\pause 

Al inicio tenemos la cadena $\lambda$: No tenemos ceros ni unos aún
\pause

\begin{tikzpicture}
	[scale=.7,auto=left]
	\tikzstyle{every state}=[scale=.6,draw=blue!50,circle,very thick,fill=blue!20]

	\node[state, initial, initial text={},]  (q0) at (1,4) {\parbox{1.5cm}{\centering $Par0$ \\$Par1$}};
\end{tikzpicture}

\pause
\textbf{Estado Final del autómata:}
\begin{block}{Recordamos:} 
Una cadena x es aceptada por un AFD si la secuencia de transiciones correspondientes a los símbolos de x conduce desde el estado inicial a un estado final.
\end{block}
\pause
En nuestro caso debemos aceptar las cadenas con cantidad par de ceros y cantidad impar de unos.
\pause
Las que cumplen esta condición son las que llegan al estado $Par0Impar1$

\begin{tikzpicture}
	[scale=.7,auto=left]
	\tikzstyle{every state}=[scale=.6,draw=blue!50,circle,very thick,fill=blue!20]

	\node[state,accepting] 			(q3) at (1,4) {\parbox{1.5cm}{\centering $Par0$ \\$Impar1$}};
\end{tikzpicture}
}

\frame{
$<Q, \Sigma, \delta, Par0Par1, F>$
\pause

$Q = \lbrace Par0Par1, Par0Impar1, Impar0Impar1, Impar0Par1 \rbrace$
\pause

$\Sigma = \lbrace 0, 1 \rbrace$
\pause

$F = \lbrace Par0Impar1 \rbrace$ 

\pause
$\delta$:

\pause

\begin{tikzpicture}
  [scale=.7,auto=left]
	\tikzstyle{every state}=[scale=.7,draw=blue!50,circle,very thick,fill=blue!20]
	\node[state, initial, initial text={},]  (q0) at (1,6) {\parbox{1.5cm}{\centering $Par0$ \\$Par1$}};
	\node[state] 						(q1) at (6,6) {\parbox{1.5cm}{\centering $Impar0$ \\$Par1$}};
	\node[state,accepting] 			(q3) at (1,1) {\parbox{1.5cm}{\centering $Par0$ \\$Impar1$}};
	\node[state] 						(q2) at (6,1) {\parbox{1.5cm}{\centering $Impar0$ \\$Impar1$}};

	\pause

	\draw [->] (q0) to [bend left=20] node [above] {$0$} (q1);
	\draw [->] (q0) to [bend left=20] node [right] {$1$} (q3);
	\pause

	\draw [->] (q1) to [bend left=20] node [above] {$0$} (q0);
	\draw [->] (q1) to [bend left=20] node [right] {$1$} (q2);
	\pause

	\draw [->] (q2) to [bend left=20] node [above] {$0$} (q3);
	\draw [->] (q2) to [bend left=20] node [left] {$1$} (q1);
	\pause

	\draw [->] (q3) to [bend left=20] node [above] {$0$} (q2);
	\draw [->] (q3) to [bend left=20] node [left] {$1$} (q0);
\end{tikzpicture}

}
