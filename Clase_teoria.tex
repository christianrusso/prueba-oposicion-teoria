

\documentclass{beamer}
\usepackage[spanish]{babel}
\usepackage[utf8]{inputenc}
\usepackage{tikz}

\usetikzlibrary{arrows,automata}

% coding: utf-8

\newcommand{\ToDo}{\color{red} {\Huge COMPLETAR} \normalcolor}
\newcommand{\ReView}{\color{red} {\Huge REVISAR!! } \normalcolor}
\newcommand{\Ask}{\color{red} {\Huge CONSULTAR!! } \normalcolor}

\newcommand{\saltosinsangria}{ \newline \newline}


\newcommand{\filaparS}[2]
{
	{#1} & {#2}\\ 
}

\newcommand{\filaimparS}[2]
{
	{#1} & {#2}\\ 
}

\newcommand{\camposcolorS}[2]
{
\textbf{{#1}} & \textbf{{#2}}\\
}



\newcommand{\filaparT}[4]
{
	\rowcolor[gray]{1}{#1} & {#2} & {#3} & {#4}\\ 
}

\newcommand{\filaimparT}[4]
{
	\rowcolor[gray]{0.8}{#1} & {#2} & {#3} & {#4}\\ 
}

\newcommand{\camposcolorT}[4]
{
	\rowcolor[cmyk]{1,1,0,0} \textbf{\color{white} {#1}} & \textbf{\color{white} {#2}} & \textbf{\color{white} {#3}} & \textbf{\color{white} {#4}} \\
}

\newcommand{\filapar}[6]
{
	\rowcolor[gray]{1}{#1} & {#2} & {#3} & {#4} & {#5} & {#6}\\ 
}

\newcommand{\filaimpar}[6]
{
	\rowcolor[gray]{0.8}{#1} & {#2} & {#3} & {#4} & {#5} & {#6}\\ 
}

\newcommand{\camposcolor}[6]
{
	\rowcolor[cmyk]{1,1,0,0} \textbf{\color{white} {#1}} & \textbf{\color{white} {#2}} & \textbf{\color{white} {#3}} & \textbf{\color{white} {#4}} & \textbf{\color{white} {#5}} & \textbf{\color{white} {#6}}\\
}

\newcommand{\imagen}[3]
{
	\begin{figure}[p!hbt]
	  \centering
	  \includegraphics[scale=#3]{../img/#1}
	  \caption{#2}
          \label{fig_#1}
	\end{figure}
} 
\newcommand{\imagenvertical}[3]
{
	\begin{figure}[p!hbt]
	  \centering
	  \includegraphics[angle=90,scale=#3]{../img/#1}
	  \caption{#2}
          \label{fig_#1}
	\end{figure}
}

%\usetheme{Berkeley}
% \usetheme{JuanLesPins}
% \usetheme{bars}
% \usetheme{split}
  \usetheme[compress]{Ilmenau}
% \mode<presentation>
%\mode<handout>%{\beamertemplatesolidbackgroundcolor{black!5}}
% \mode<article>{\usepackage{fullpage}}
% \usepackage{pgfpages}
%\pgfpagelayout{resize}[a4paper,border shrink=5mm,landscape]

\include{epsfig}
\title{Autómatas finitos determinísticos}

\author{Christian Sebastian Russo}
\date{16 de Septiembre de 2015}

\begin{document}

\frame{\titlepage}

\frame{\tableofcontents}

\section{Autómatas finitos determinísticos}
\subsection{Definición}

\frame{
\frametitle{Autómata finito determinístico - AFD}
Se define mediante la 5-upla $<Q, \Sigma, \delta, q_0, F>$ donde 
\begin{itemize}
\item $Q$ es un conjunto finito de \textbf{estados }
\item $\Sigma$ es el conjunto finito de símbolos correspondiente al \textbf{alfabeto de entrada}
\item $\delta : Q \times \Sigma \rightarrow Q$ es la \textbf{función de transición}, que indica las transiciones entre los distintos estados
\item $q_0 \in Q$ es el \textbf{estado inicial}
\item $F \subseteq Q$ es el conjunto de \textbf{estados finales}
\end{itemize}
\textbf{Determinístico:} para todo estado del autómata existe como máximo una transición definida para cada símbolo del alfabeto 
}

\subsection{Conceptos}

\frame{
\frametitle{Relación entre cadenas y autómatas}
\begin{block}{Función de transición generalizada}
	La función de transición propuesta anteriormente, puede extenderse para que en lugar de	\textbf{símbolos} del alfabeto acepte \textbf{cadenas} en ese alfabeto, es decir:
	$\widehat{\delta} : Q \times \Sigma^* \rightarrow Q$
	\begin{itemize}
		\item $\widehat{\delta} (q, \lambda) = q$
		\item $\widehat{\delta} (q, xa) = \delta ( \widehat{\delta} (q, x), a)$ con $x \in \Sigma^*$ y $a \in \Sigma$
	\end{itemize}
\end{block}
\begin{block}{Cadena aceptada por un AFD}
	Una cadena x es aceptada por un AFD $M = <Q, \Sigma, \delta, q_0, F>$ sii $\widehat{\delta} (q_0, x) \in F$, es decir, si la secuencia de transiciones correspondientes a los símbolos de la cadena x conduce desde el estado inicial a un estado final.
\end{block}
}


\section{Ejercicio}
\subsection{Enunciado}
\frame{
\frametitle{Ejercicio 1-c) (Práctica 1 - Teoría de lenguajes)}
Construir un autómata finito para el siguiente lenguaje:

Cadenas sobre $\Sigma = \lbrace 0, 1 \rbrace$ con cantidad par de ceros y cantidad impar de unos.
}

\subsection{Consideraciones}
\frame{
Existen cuatro posibles combinaciones de paridades de ceros y unos para un número binario:
\begin{itemize}
\item Cantidad par de ceros, cantidad par de unos. 
\pause

Ej: $\lambda$, 0011, 1111, 010010 
\pause
\item Cantidad impar de ceros, cantidad par de unos. 
\pause

Ej: 011, 11011, 0
\pause
\item Cantidad impar de ceros, cantidad impar de unos.
\pause

Ej: 01, 1110, 0001
\pause
\item Cantidad par de ceros, cantidad impar de unos. 
\pause

Ej: 001, 111, 01000
\end{itemize}
De estas cuatro posibles combinaciones, nuestro lenguaje acepta sólo la última.
}

\subsection{Solución}
\frame{
Podemos representar las cuatro distintas combinaciones mencionadas anteriormente, como cuatro estados distintos:
\begin{itemize}
\item $Par0Par1$: Cantidad par de ceros, cantidad par de unos. 
\item $Impar0Par1$: Cantidad impar de ceros, cantidad par de unos. 
\item $Impar0Impar1$: Cantidad impar de ceros, cantidad impar de unos.
\item $Par0Impar1$: Cantidad par de ceros, cantidad impar de unos. 
\end{itemize}
}

\frame{
\textbf{Estado Inicial del autómata:}
\pause 

Al inicio tenemos la cadena $\lambda$: No tenemos ceros ni unos aún
\pause

\begin{tikzpicture}
	[scale=.7,auto=left]
	\tikzstyle{every state}=[scale=.6,draw=blue!50,circle,very thick,fill=blue!20]

	\node[state, initial, initial text={},]  (q0) at (1,4) {\parbox{1.5cm}{\centering $Par0$ \\$Par1$}};
\end{tikzpicture}

\pause
\textbf{Estado Final del autómata:}
\begin{block}{Recordamos:} 
Una cadena x es aceptada por un AFD si la secuencia de transiciones correspondientes a los símbolos de x conduce desde el estado inicial a un estado final.
\end{block}
\pause
En nuestro caso debemos aceptar las cadenas con cantidad par de ceros y cantidad impar de unos.
\pause
Las que cumplen esta condición son las que llegan al estado $Par0Impar1$

\begin{tikzpicture}
	[scale=.7,auto=left]
	\tikzstyle{every state}=[scale=.6,draw=blue!50,circle,very thick,fill=blue!20]

	\node[state,accepting] 			(q3) at (1,4) {\parbox{1.5cm}{\centering $Par0$ \\$Impar1$}};
\end{tikzpicture}
}

\frame{
$<Q, \Sigma, \delta, Par0Par1, F>$
\pause

$Q = \lbrace Par0Par1, Par0Impar1, Impar0Impar1, Impar0Par1 \rbrace$
\pause

$\Sigma = \lbrace 0, 1 \rbrace$
\pause

$F = \lbrace Par0Impar1 \rbrace$ 

\pause
$\delta$:

\pause

\begin{tikzpicture}
  [scale=.7,auto=left]
	\tikzstyle{every state}=[scale=.7,draw=blue!50,circle,very thick,fill=blue!20]
	\node[state, initial, initial text={},]  (q0) at (1,6) {\parbox{1.5cm}{\centering $Par0$ \\$Par1$}};
	\node[state] 						(q1) at (6,6) {\parbox{1.5cm}{\centering $Impar0$ \\$Par1$}};
	\node[state,accepting] 			(q3) at (1,1) {\parbox{1.5cm}{\centering $Par0$ \\$Impar1$}};
	\node[state] 						(q2) at (6,1) {\parbox{1.5cm}{\centering $Impar0$ \\$Impar1$}};

	\pause

	\draw [->] (q0) to [bend left=20] node [above] {$0$} (q1);
	\draw [->] (q0) to [bend left=20] node [right] {$1$} (q3);
	\pause

	\draw [->] (q1) to [bend left=20] node [above] {$0$} (q0);
	\draw [->] (q1) to [bend left=20] node [right] {$1$} (q2);
	\pause

	\draw [->] (q2) to [bend left=20] node [above] {$0$} (q3);
	\draw [->] (q2) to [bend left=20] node [left] {$1$} (q1);
	\pause

	\draw [->] (q3) to [bend left=20] node [above] {$0$} (q2);
	\draw [->] (q3) to [bend left=20] node [left] {$1$} (q0);
\end{tikzpicture}

}


\section{Bibliografía}
\frame{
\frametitle{Bibliografía}
\begin{thebibliography}{10}
	\bibitem{}
		Introduction to Automata theory, languages, and computation
	\newblock \emph{Hopcroft, John E. y Ullman, Jeffrey D.}
\end{thebibliography}
}


\end{document}
